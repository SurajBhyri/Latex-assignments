\documentclass[12pt]{article}
\usepackage{amsmath}
\newcommand{\myvec}[1]{\ensuremath{\begin{pmatrix}#1\end{pmatrix}}}
\newcommand{\mydet}[1]{\ensuremath{\begin{vmatrix}#1\end{vmatrix}}}
\newcommand{\solution}{\noindent \textbf{Solution: }}
\providecommand{\brak}[1]{\ensuremath{\left(#1\right)}}
\providecommand{\norm}[1]{\left\lVert#1\right\rVert}
\let\vec\mathbf

\title{Pair of linear equation in two variables}
\author{Saipreet Pattjoshi (spattjoshi@sriprakashschools.com)}

\begin{document}
\maketitle
\section*{10$^{th}$ Maths - Chapter 3}
\begin{enumerate}
\item  On comparing $\frac{a_1}{a_2}$, $\frac{b_1}{b_2}$, and $\frac{c_1}{c_2} $
Find out whether the following pair of linear equation are consistent or inconsistent\\
\end{enumerate}
\solution \\
Given Data:\\
$\frac{3}{2}x + \frac{5}{3}y=7,\\$
$\implies 6 \times \frac{3}{2}x + \frac{5}{3}y=7$,
9x-10y=14\\
This can also be written as:
\begin{align}
\myvec{9&10&14\\9&-10&14}
\end{align}
$R_2 \xrightarrow\ R_1 + R_2$\\ 
we get,
\begin{align}
\myvec{9&10&14\\0&20&28}\\
\end{align}
$R_2 \xrightarrow\ \frac{R_2}{4}$
\begin{align}
\myvec{9&10&14\\0&5&7}\\
\end{align}
$R_1 \xrightarrow\ R_1+2R_2$
\begin{align}
\myvec{9&0&28\\0&5&7}\\
\end{align}
$R_2 \xrightarrow\ R_2/5 ; R_1\xrightarrow\ \frac{R_1}{9}$\\
\begin{align}
\myvec{1&0&\frac{28}{9}\\0&1&\frac{7}{5}}
\end{align}
Since, The values of $\frac{a_1}{a_2} \neq \frac{b_1}{b_2}$\\
It is a consistent equation
\end{document}