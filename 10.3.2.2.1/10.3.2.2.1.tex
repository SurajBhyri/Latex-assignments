\documentclass[10pt]{article}
\usepackage{amsmath}
\newcommand{\myvec}[1]{\ensuremath{\begin{pmatrix}#1\end{pmatrix}}}
\newcommand{\mydet}[1]{\ensuremath{\begin{vmatrix}#1\end{vmatrix}}}
\newcommand{\solution}{\noindent \textbf{Solution: }}
\providecommand{\brak}[1]{\ensuremath{\left(#1\right)}}
\providecommand{\norm}[1]{\left\lVert#1\right\rVert}
\let\vec\mathbf

\title{Linear Equations In Two Variables}
\author{P.Kishan (pusarlakishan@sriprakashschools.com)}

\begin{document}
\maketitle
\section*{Class 10$^{th}$ Maths - Chapter 3}
This is Problem-2.1 from Exercise 3.2
\begin{enumerate}
\item On comparing the ratios $\frac{a_1}{a_2}$,$\frac{b_1}{b_2}$ and $\frac{c_1}{c_2}$, find out wether the lines representing the following pair of linear equations intersect at a point, parallel and coincident.\\
\begin{align} 
5x-4y+8=0\\
7x+6y-9=0
\end{align}
solution \\
this can be written as \\
\begin{align}
\myvec{5&-4\\7&6}\myvec{x\\y}=\myvec{-8\\9}\\   
x=\frac{\mydet{ \vec{b} & \vec{a_2}}}{\mydet{ \vec{a_1} &\vec{a_2} }}=
\frac{\mydet{-8&-4 \\ 9 & 6 }}{\mydet{5&-4\\7&6}} =
\frac{(-8)(6)-(9)(-4)}{(5)(6)-(7)(-4)} =
\frac{-48+36}{30+28}=\frac{-12}{58}\\
y=\frac{\mydet{\vec{a_1}&\vec{b}}}{\mydet{\vec{a_1}&\vec{a_2}}} =
\frac{\mydet{5&-8\\7&9}}{\mydet{5&-4\\7&6}}=
\frac{(5)(9)-(7)(-8)}{(5)(6)-(7)(-4)}=
\frac{45+56}{30+28}=\frac{101}{58} 
\end{align}
\end{enumerate}
\end{document}
