\documentclass[12pt]{article}
\usepackage{amsmath}
\newcommand{\myvec}[1]{\ensuremath{\begin{pmatrix}#1\end{pmatrix}}}
\newcommand{\mydet}[1]{\ensuremath{\begin{vmatrix}#1\end{vmatrix}}}
\newcommand{\solution}{\noindent \textbf{Solution: }}
\providecommand{\brak}[1]{\ensuremath{\left(#1\right)}}
\providecommand{\norm}[1]{\left\lVert#1\right\rVert}
\let\vec\mathbf
\title{Linear Equations in Two Variables}
\author{harshita (paidisettyharshita@sriprakashschools.com)}
\begin{document}
\maketitle
\section*{10$^{th}$ Maths - Chapter 3}
This is Problem-4.1 from Exercise 3.2
\begin{enumerate}
\item On comparing the ratios $\frac{a_1}{a_2}$ , $\frac{b_1}{b_2}$ ,$\frac{c_1}{c_2}$, find out whether the lines representing the following pairs of linear equations intersect at a point, are parallel or coincident:\\
5x-3y=11\\ 
-10x+6y=22\\
\end{enumerate}
\solution\\
This can also be written as:
\begin{align}
\myvec{5&-3&11\\-10&6&22}
\end{align}
now,Making $R_2 \xrightarrow\ 2R_1 - R_2$\\ 
we get,
\begin{align}
\myvec{5&-3&11\\0&0&0}
\end{align}
Since, we are getting zero in $R_2$\\
It is a dependent equation.
\end{document}
