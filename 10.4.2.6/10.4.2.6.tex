
\documentclass[12pt]{article}
\usepackage{amsmath}
\newcommand{\myvec}[1]{\ensuremath{\begin{pmatrix}#1\end{pmatrix}}}
\newcommand{\mydet}[1]{\ensuremath{\begin{vmatrix}#1\end{vmatrix}}}
\newcommand{\solution}{\noindent \textbf{Solution: }}
\providecommand{\brak}[1]{\ensuremath{\left(#1\right)}}
\providecommand{\norm}[1]{\left\lVert#1\right\rVert}
\let\vec\mathbf

\title{Quadratic equations}
\author{UPPADAJASWANTH(UPPADAJASWANTH@SRIPRAKASHSCHOOLS.COM)}
\begin{document}
\section*{10$^{th}$ Maths - Chapter 4}
This is Problem-6(v) from Exercise 4.2\\
A cottage industry produces a certain number of pottery articles in a day. It was observed
on a particular day that the cost of production of each article (in rupees) was 3 more than
twice the number of articles produced on that day. If the total cost of production on that
day was ` 90, find the number of articles produced and the cost of each article\\
Given:
Let the number of these articles produced in a day be = x.\\
Cost of each article was 3 more than twice the number of articles
produced that be = 3 + 2x\\
total cost of the production = 90

\solution\\
Given:
Let the number of these articles produced in a day be = x.\\
Cost of each article was 3 more than twice the number of articles
produced that be = 3 + 2x\\
total cost of the production = 90\\
\begin{align}
{(x)(2x + 3) = 90}\\=&
{2x^2+3x=90}\\=&
{2x^2+3x-90=0}\\=&
{2x^2-12x+15x-90}\\=&
{2x(x-6)+15(x-6)}\\=&
{(2x+15)(x-6)}\\=&
{x=\frac{-15}{2}}\\
or\\
{x=6}\\
{3+2(6)=15}\\
\end{align}
Therefore, cost of each article = 15\\
Number of articles produced that day = 6\\


\end{document}
