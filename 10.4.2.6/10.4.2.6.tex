\documentclass[12pt]{article}
\usepackage{amsmath}
\newcommand{\myvec}[1]{\ensuremath{\begin{pmatrix}#1\end{pmatrix}}}
\newcommand{\mydet}[1]{\ensuremath{\begin{vmatrix}#1\end{vmatrix}}}
\newcommand{\solution}{\noindent \textbf{Solution: }}
\providecommand{\brak}[1]{\ensuremath{\left(#1\right)}}
\providecommand{\norm}[1]{\left\lVert#1\right\rVert}
\let\vec\mathbf

\title{Quadratic-Equations}
\author{UPPADAJASWANTH (UPPADAJASWANTH@SRIPRAKASHSCHOOLS.COM)}


\begin{document}
\maketitle
\section*{10$^{th}$ Maths - Chapter 4}
This is Problem-6(v) from Exercise 4.2\\
\begin{enumerate}
\item A cottage industry produces a certain number of pottery articles in a day. It was observed
on a particular day that the cost of production of each article (in rupees) was 3 more than
twice the number of articles produced on that day. If the total cost of production on that
day was 90, find the number of articles produced and the cost of each article\\
\end{enumerate}
\solution \\
Given:
Let the number of these articles produced in a day be = x.\\
Cost of each article was 3 more than twice the number of articles
produced that be = 3 + 2x\\
total cost of the production = 90\\
\begin{align}
(x)(2x + 3) &= 90\\
2x^2+3x-90&=0\\
\end{align}
Using formula method,first solution is:\\
\begin{align}
x_1 &= \frac{-b+\sqrt{b^2-4ac}}{2a}\\
&= \frac{-(3)+\sqrt{(3)^2-4(2)(-90)}}{2(2)}\\
&= \frac{-3+\sqrt{9+720}}{4}\\
&= \frac{-3+\sqrt{729}}{4}\\
&= \frac{-3+27}{4}\\
&= \frac{24}{4}=6
\end{align}
the second solution is:\\
\begin{align}
x_2 &= \frac{-b-\sqrt{b^2-4ac}}{2a}\\
&= \frac{-(3)-\sqrt{729}}{4}\\
&= \frac{-30}{4}\\
&= \frac{-15}{2}
\end{align}

Since the number of articles cannot be negative, the number of articles produced that day = 6\\
Therefore, cost of each article = 15\\
\end{document}
