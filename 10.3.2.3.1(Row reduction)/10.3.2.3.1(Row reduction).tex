\documentclass[10pt]{article}
\usepackage{amsmath}
\newcommand{\myvec}[1]{\ensuremath{\begin{pmatrix}#1\end{pmatrix}}}
\newcommand{\mydet}[1]{\ensuremath{\begin{vmatrix}#1\end{vmatrix}}}
\newcommand{\solution}{\noindent \textbf{Solution: }}
\providecommand{\brak}[1]{\ensuremath{\left(#1\right)}}
\providecommand{\norm}[1]{\left\lVert#1\right\rVert}
\let\vec\mathbf

\title{Pair of Linear Equations in Two Variables}
\author{sai charvi (patnanasaicharvi@sriprakashschools.com)}


\begin{document}
\maketitle
\section*{Class 10$^{th}$ Maths - Chapter 3}
This is Problem-1.1 from Exercise 3.2
\begin{enumerate}
\item 3x+2y=5 and 2x-3y=7

\solution
\begin{align}
3x+2y=5\\
2x-3y=7\\
\end{align}

it can be written as\\
\begin{align}
\myvec{3&2&5\\2&-3&7}\\
R_1\xrightarrow\ 2R_1-3R_2\\
\myvec{0&13&-11\\2&-3&7}\\
R_1\xrightarrow\ \frac{R_1}{13}\\
\myvec{0&1&\frac{-11}{13}\\2&-3&7}\\
R_2\xrightarrow\ R_2+3R_1\\
\myvec{0&1&\frac{-11}{13}\\2&0&\frac{58}{13}}\\
R_2\xrightarrow\ \frac{R_2}{2}\\
\myvec{0&1&\frac{-11}{13}\\1&0&\frac{29}{13}}\\
\end{align}
therefore y=-11/13 and x=29/8
\end{enumerate}
\end{document}
